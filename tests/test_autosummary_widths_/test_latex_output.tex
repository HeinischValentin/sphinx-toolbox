<%- set open_smart_quote = "\\textquotesingle{}" if (3, 5) > sphinx_version >= (4, 2) else "‘" -%>
<%- set close_smart_quote = "\\textquotesingle{}" if (3, 5) > sphinx_version >= (4, 2) else "’" -%>
<%- set open_smart_quote_b = "\\textquotesingle{}" if sphinx_version >= (4, 2) else "‘" -%>
<%- set close_smart_quote_b = "\\textquotesingle{}" if sphinx_version >= (4, 2) else "’" -%>
<%- set hline = "\\sphinxhline" if sphinx_version >= (5, 3) else "\\hline" -%>
<%- set sphinxparam = "sphinxparam" if sphinx_version >= (6, 2) else "emph" -%>
<%- if sphinx_version >= (7, 1) -%>
<%- set sphinxparamcomma = "\\sphinxparamcomma \\sphinxparam" -%>
<%- elif sphinx_version >= (6, 2) -%>
<%- set sphinxparamcomma = ", \\sphinxparam" -%>
<%- else -%>
<%- set sphinxparamcomma = ", \\emph" -%>
<%- endif -%>
<%- macro sphinx6x_2nd(char) -%>
<% if sphinx_version >= (6, 2) %>,<<char>><% endif %>
<%- endmacro -%>
%% Generated by Sphinx.
\def\sphinxdocclass{report}
\documentclass[letterpaper,10pt,english]{sphinxmanual}
\ifdefined\pdfpxdimen
   \let\sphinxpxdimen\pdfpxdimen\else\newdimen\sphinxpxdimen
\fi \sphinxpxdimen=.75bp\relax
<% if sphinx_version >= (4, 0) %>\ifdefined\pdfimageresolution
    \pdfimageresolution= \numexpr \dimexpr1in\relax/\sphinxpxdimen\relax
\fi
%% let collapsable pdf bookmarks panel have high depth per default
\PassOptionsToPackage{bookmarksdepth=5}{hyperref}<% if sphinx_version >= (5, 3) %>
<% endif %><% if sphinx_version >= (6, 0) %>
\PassOptionsToPackage{booktabs}{sphinx}
\PassOptionsToPackage{colorrows}{sphinx}<% endif %>
<% endif %>
\PassOptionsToPackage{warn}{textcomp}
\usepackage[utf8]{inputenc}
\ifdefined\DeclareUnicodeCharacter
% support both utf8 and utf8x syntaxes
  \ifdefined\DeclareUnicodeCharacterAsOptional
    \def\sphinxDUC#1{\DeclareUnicodeCharacter{"#1}}
  \else
    \let\sphinxDUC\DeclareUnicodeCharacter
  \fi
  \sphinxDUC{00A0}{\nobreakspace}
  \sphinxDUC{2500}{\sphinxunichar{2500}}
  \sphinxDUC{2502}{\sphinxunichar{2502}}
  \sphinxDUC{2514}{\sphinxunichar{2514}}
  \sphinxDUC{251C}{\sphinxunichar{251C}}
  \sphinxDUC{2572}{\textbackslash}
\fi
\usepackage{cmap}
\usepackage[T1]{fontenc}
\usepackage{amsmath,amssymb,amstext}
\usepackage{babel}

<%+ if sphinx_version >= (4, 0) %>

\usepackage{tgtermes}
\usepackage{tgheros}
\renewcommand{\ttdefault}{txtt}

<% else %>

\usepackage{times}
\expandafter\ifx\csname T@LGR\endcsname\relax
\else
% LGR was declared as font encoding
  \substitutefont{LGR}{\rmdefault}{cmr}
  \substitutefont{LGR}{\sfdefault}{cmss}
  \substitutefont{LGR}{\ttdefault}{cmtt}
\fi
\expandafter\ifx\csname T@X2\endcsname\relax
  \expandafter\ifx\csname T@T2A\endcsname\relax
  \else
  % T2A was declared as font encoding
    \substitutefont{T2A}{\rmdefault}{cmr}
    \substitutefont{T2A}{\sfdefault}{cmss}
    \substitutefont{T2A}{\ttdefault}{cmtt}
  \fi
\else
% X2 was declared as font encoding
  \substitutefont{X2}{\rmdefault}{cmr}
  \substitutefont{X2}{\sfdefault}{cmss}
  \substitutefont{X2}{\ttdefault}{cmtt}
\fi
<% endif %>

\usepackage[Bjarne]{fncychap}
\usepackage{sphinx}
<% if sphinx_version >= (4, 0) %>
\fvset{fontsize=auto}
<% else %>
\fvset{fontsize=\small}
<% endif -%>
\usepackage{geometry}


% Include hyperref last.
\usepackage{hyperref}
% Fix anchor placement for figures with captions.
\usepackage{hypcap}% it must be loaded after hyperref.
% Set up styles of URL: it should be placed after hyperref.
\urlstyle{same}


\usepackage{sphinxmessages}



\newcommand\thesymbolfootnote{\fnsymbol{footnote}}\let\thenumberfootnote\thefootnote

\makeatletter
\newcolumntype{\Xx}[2]{>{\raggedright\arraybackslash}p{\dimexpr
    (\linewidth-\arrayrulewidth)*#1/#2-\tw@\tabcolsep-\arrayrulewidth\relax}}
\makeatother


\title{Python}
\date{Mar 11, 2021}
\release{}
\author{unknown}
\newcommand{\sphinxlogo}{\vbox{}}
\renewcommand{\releasename}{}
\makeindex
\begin{document}

<% if sphinx_version >= (5, 0) %>\ifdefined\shorthandoff
  \ifnum\catcode`\=\string=\active\shorthandoff{=}\fi
  \ifnum\catcode`\"=\active\shorthandoff{"}\fi
\fi

<% endif %>\pagestyle{empty}
\sphinxmaketitle
\pagestyle{plain}
\sphinxtableofcontents
\pagestyle{normal}
\phantomsection\label{\detokenize{index::doc}}

\index{module@\spxentry{module}!string@\spxentry{string}}\index{string@\spxentry{string}!module@\spxentry{module}}
A collection of string constants.

Public module variables:

whitespace \textendash{} a string containing all ASCII whitespace
ascii\_lowercase \textendash{} a string containing all ASCII lowercase letters
ascii\_uppercase \textendash{} a string containing all ASCII uppercase letters
ascii\_letters \textendash{} a string containing all ASCII letters
digits \textendash{} a string containing all ASCII decimal digits
hexdigits \textendash{} a string containing all ASCII hexadecimal digits
octdigits \textendash{} a string containing all ASCII octal digits
punctuation \textendash{} a string containing all ASCII punctuation characters
printable \textendash{} a string containing all ASCII characters considered printable


\vspace{10px}


\sphinxstylestrong{Classes:}


\vspace{-5px}



\begin{savenotes}<% if sphinx_version >= (5, 3) %>
\sphinxatlongtablestart
\sphinxthistablewithglobalstyle
\sphinxthistablewithnovlinesstyle
<% endif %><% if sphinx_version >= (6, 2) %>\makeatletter
  \LTleft \@totalleftmargin plus1fill
  \LTright\dimexpr\columnwidth-\@totalleftmargin-\linewidth\relax plus1fill
\makeatother
\begin{longtable}{\Xx{3}{10}\Xx{7}{10}}<% elif sphinx_version >= (5, 3) %>\begin{longtable}[c]{\Xx{3}{10}\Xx{7}{10}}<% else %>\sphinxatlongtablestart\begin{longtable}[c]{\Xx{3}{10}\Xx{7}{10}}<% endif %><% if sphinx_version >= (5, 3) %>
\sphinxtoprule
<% else %>
\hline

<% endif %>\endfirsthead

\multicolumn{2}{c}<% if sphinx_version >= (5, 3) %>{\sphinxnorowcolor
    \makebox[0pt]<% else %>%
{\makebox[0pt]<% endif %>{\sphinxtablecontinued{\tablename\ \thetable{} \textendash{} continued from previous page}}<% if sphinx_version >= (5, 3) %>%
<% endif %>}\\
<% if sphinx_version >= (5, 3) %>\sphinxtoprule
<% else %>\hline

<% endif %>\endhead

<% if sphinx_version >= (5, 3) %>\sphinxbottomrule<% else %>\hline<% endif%>
\multicolumn{2}{r}{<% if sphinx_version >= (5, 3) %>\sphinxnorowcolor
    <% endif %>\makebox[0pt][r]{\sphinxtablecontinued{continues on next page}}<% if sphinx_version >= (5, 3) %>%
<% endif %>}\\
\endfoot

\endlastfoot<% if sphinx_version >= (5, 3) %>
\sphinxtableatstartofbodyhook<% endif%>

{\hyperref[\detokenize{index:string.Template}]{\sphinxcrossref{\sphinxcode{\sphinxupquote{Template}}}}}(template)
&
A string class for supporting \$\sphinxhyphen{}substitutions.
\\
<% if sphinx_version >= (5, 3) %>\sphinxbottomrule<% else %>\hline<% endif %>
\end{longtable}<% if sphinx_version >= (5, 3) %>
\sphinxtableafterendhook
<% endif %>\sphinxatlongtableend<% if sphinx_version >= (5, 3) %>
<% endif %>\end{savenotes}

\sphinxstylestrong{Functions:}


\vspace{-5px}



\begin{savenotes}<% if sphinx_version >= (5, 3) %>
<% endif %>\sphinxatlongtablestart<% if sphinx_version >= (5, 3) %>
\sphinxthistablewithglobalstyle
\sphinxthistablewithnovlinesstyle
<% endif %><% if sphinx_version >= (6, 2) %>\makeatletter
  \LTleft \@totalleftmargin plus1fill
  \LTright\dimexpr\columnwidth-\@totalleftmargin-\linewidth\relax plus1fill
\makeatother
\begin{longtable}{\Xx{3}{10}\Xx{7}{10}}<% else %>\begin{longtable}[c]{\Xx{3}{10}\Xx{7}{10}}<% endif %><% if sphinx_version >= (5, 3) %>
\sphinxtoprule
<% else %>
\hline

<% endif %>\endfirsthead

\multicolumn{2}{c}<% if sphinx_version >= (5, 3) %>{\sphinxnorowcolor
    \makebox[0pt]<% else %>%
{\makebox[0pt]<% endif %>{\sphinxtablecontinued{\tablename\ \thetable{} \textendash{} continued from previous page}}<% if sphinx_version >= (5, 3) %>%
<% endif %>}\\
<% if sphinx_version >= (5, 3) %>\sphinxtoprule
<% else %>\hline

<% endif %>\endhead

<% if sphinx_version >= (5, 3) %>\sphinxbottomrule<% else %>\hline<% endif%>
\multicolumn{2}{r}{<% if sphinx_version >= (5, 3) %>\sphinxnorowcolor
    <% endif %>\makebox[0pt][r]{\sphinxtablecontinued{continues on next page}}<% if sphinx_version >= (5, 3) %>%
<% endif %>}\\
\endfoot

\endlastfoot<% if sphinx_version >= (5, 3) %>
\sphinxtableatstartofbodyhook<% endif%>

{\hyperref[\detokenize{index:string.capwords}]{\sphinxcrossref{\sphinxcode{\sphinxupquote{capwords}}}}}(s {[},sep{]})
&
Split the argument into words using split, capitalize each word using capitalize, and join the capitalized words using join.
\\
<% if sphinx_version >= (5, 3) %>\sphinxbottomrule<% else %>\hline<% endif %>
\end{longtable}<% if sphinx_version >= (5, 3) %>
\sphinxtableafterendhook
<% endif %>\sphinxatlongtableend<% if sphinx_version >= (5, 3) %>
<% endif %>\end{savenotes}
\index{Template (class in string)@\spxentry{Template}\spxextra{class in string}}

\begin{fulllineitems}
\phantomsection\label{\detokenize{index:string.Template}}
<%- if sphinx_version > (4, 5) %>
\pysigstartsignatures
<% endif -%>
\pysiglinewithargsret{\sphinxbfcode{\sphinxupquote{class<% if sphinx_version >= (4, 3) %>\DUrole{w<<sphinx6x_2nd('w')>>}{  }<% else %> <% endif %>}}\sphinxcode{\sphinxupquote{string.}}\sphinxbfcode{\sphinxupquote{Template}}}{\<< sphinxparam >>{\DUrole{n<<sphinx6x_2nd("n")>>}{template}}}{}
<% if sphinx_version > (4, 5) %>\pysigstopsignatures
<% endif -%>
A string class for supporting \$\sphinxhyphen{}substitutions.

\end{fulllineitems}

\index{capwords() (in module string)@\spxentry{capwords()}\spxextra{in module string}}

\begin{fulllineitems}
\phantomsection\label{\detokenize{index:string.capwords}}
<%- if sphinx_version > (4, 5) %>
\pysigstartsignatures
<% endif -%>
\pysiglinewithargsret{\sphinxcode{\sphinxupquote{string.}}\sphinxbfcode{\sphinxupquote{capwords}}}{\<<sphinxparam>>{<% if sphinx_version >= (4, 3)%>\DUrole{n<< sphinx6x_2nd('n') >>}{s}}<% else %>s}<% endif %>\sphinxoptional{<<sphinxparamcomma>>{<% if sphinx_version >= (4, 3)%>\DUrole{n<<sphinx6x_2nd("n")>>}{sep}}<% else %>sep}<% endif %>}}{{
<%- if sphinx_version >= (4, 4) %> $\rightarrow$ string}}
<%- else %> $\rightarrow$ {\hyperref[\detokenize{index:module-string}]{\sphinxcrossref{string}}}}}
<%- endif -%>
<%- if sphinx_version > (4, 5) %>
\pysigstopsignatures
<%- endif %>
Split the argument into words using split, capitalize each
word using capitalize, and join the capitalized words using
join.  If the optional second argument sep is absent or None,
runs of whitespace characters are replaced by a single space
and leading and trailing whitespace are removed, otherwise
sep is used to split and join the words.

\end{fulllineitems}

\phantomsection\label{\detokenize{index:module-textwrap}}\index{module@\spxentry{module}!textwrap@\spxentry{textwrap}}\index{textwrap@\spxentry{textwrap}!module@\spxentry{module}}
Text wrapping and filling.


\vspace{10px}


\sphinxstylestrong{Classes:}


\vspace{-5px}



\begin{savenotes}<% if sphinx_version >= (5, 3) %>
<% endif %>\sphinxatlongtablestart<% if sphinx_version >= (5, 3) %>
\sphinxthistablewithglobalstyle
\sphinxthistablewithnovlinesstyle
<% endif %><% if sphinx_version >= (6, 2) %>\makeatletter
  \LTleft \@totalleftmargin plus1fill
  \LTright\dimexpr\columnwidth-\@totalleftmargin-\linewidth\relax plus1fill
\makeatother
\begin{longtable}{\Xx{4}{10}\Xx{60}{100}}<% else %>\begin{longtable}[c]{\Xx{4}{10}\Xx{60}{100}}<% endif %><% if sphinx_version >= (5, 3) %>
\sphinxtoprule
<% else %>
\hline

<% endif %>\endfirsthead

\multicolumn{2}{c}<% if sphinx_version >= (5, 3) %>{\sphinxnorowcolor
    \makebox[0pt]<% else %>%
{\makebox[0pt]<% endif %>{\sphinxtablecontinued{\tablename\ \thetable{} \textendash{} continued from previous page}}<% if sphinx_version >= (5, 3) %>%
<% endif %>}\\
<% if sphinx_version >= (5, 3) %>\sphinxtoprule
<% else %>\hline

<% endif %>\endhead

<% if sphinx_version >= (5, 3) %>\sphinxbottomrule<% else %>\hline<% endif%>
\multicolumn{2}{r}{<% if sphinx_version >= (5, 3) %>\sphinxnorowcolor
    <% endif %>\makebox[0pt][r]{\sphinxtablecontinued{continues on next page}}<% if sphinx_version >= (5, 3) %>%
<% endif %>}\\
\endfoot

\endlastfoot<% if sphinx_version >= (5, 3) %>
\sphinxtableatstartofbodyhook<% endif%>

{\hyperref[\detokenize{index:textwrap.TextWrapper}]{\sphinxcrossref{\sphinxcode{\sphinxupquote{TextWrapper}}}}}({[}width, initial\_indent, <% if sphinx_version >= (4, 2) %>...<% else %>…<% endif %>{]})
&
Object for wrapping/filling text.
\\
<% if sphinx_version >= (5, 3) %>\sphinxbottomrule<% else %>\hline<% endif %>
\end{longtable}<% if sphinx_version >= (5, 3) %>
\sphinxtableafterendhook
<% endif %>\sphinxatlongtableend<% if sphinx_version >= (5, 3) %>
<% endif %>\end{savenotes}

\sphinxstylestrong{Functions:}


\vspace{-5px}



\begin{savenotes}<% if sphinx_version >= (5, 3) %>
<% endif %>\sphinxatlongtablestart<% if sphinx_version >= (5, 3) %>
\sphinxthistablewithglobalstyle
\sphinxthistablewithnovlinesstyle
<% endif %><% if sphinx_version >= (6, 2) %>\makeatletter
  \LTleft \@totalleftmargin plus1fill
  \LTright\dimexpr\columnwidth-\@totalleftmargin-\linewidth\relax plus1fill
\makeatother
\begin{longtable}{\Xx{4}{10}\Xx{60}{100}}<% else %>\begin{longtable}[c]{\Xx{4}{10}\Xx{60}{100}}<% endif %><% if sphinx_version >= (5, 3) %>
\sphinxtoprule
<% else %>
\hline

<% endif %>\endfirsthead

\multicolumn{2}{c}<% if sphinx_version >= (5, 3) %>{\sphinxnorowcolor
    \makebox[0pt]<% else %>%
{\makebox[0pt]<% endif %>{\sphinxtablecontinued{\tablename\ \thetable{} \textendash{} continued from previous page}}<% if sphinx_version >= (5, 3) %>%
<% endif %>}\\
<% if sphinx_version >= (5, 3) %>\sphinxtoprule
<% else %>\hline

<% endif %>\endhead

<% if sphinx_version >= (5, 3) %>\sphinxbottomrule<% else %>\hline<% endif%>
\multicolumn{2}{r}{<% if sphinx_version >= (5, 3) %>\sphinxnorowcolor
    <% endif %>\makebox[0pt][r]{\sphinxtablecontinued{continues on next page}}<% if sphinx_version >= (5, 3) %>%
<% endif %>}\\
\endfoot

\endlastfoot<% if sphinx_version >= (5, 3) %>
\sphinxtableatstartofbodyhook<% endif%>

{\hyperref[\detokenize{index:textwrap.dedent}]{\sphinxcrossref{\sphinxcode{\sphinxupquote{dedent}}}}}(text)
&
Remove any common leading whitespace from every line in \sphinxtitleref{text}.
\\
<< hline >>
{\hyperref[\detokenize{index:textwrap.fill}]{\sphinxcrossref{\sphinxcode{\sphinxupquote{fill}}}}}(text{[}, width{]})
&
Fill a single paragraph of text, returning a new string.
\\
<< hline >>
{\hyperref[\detokenize{index:textwrap.indent}]{\sphinxcrossref{\sphinxcode{\sphinxupquote{indent}}}}}(text, prefix{[}, predicate{]})
&
Adds <<open_smart_quote_b>>prefix<<close_smart_quote_b>> to the beginning of selected lines in <<open_smart_quote_b>>text<<close_smart_quote_b>>.
\\
<< hline >>
{\hyperref[\detokenize{index:textwrap.shorten}]{\sphinxcrossref{\sphinxcode{\sphinxupquote{shorten}}}}}(text, width, **kwargs)
&
Collapse and truncate the given text to fit in the given width.
\\
<< hline >>
{\hyperref[\detokenize{index:textwrap.wrap}]{\sphinxcrossref{\sphinxcode{\sphinxupquote{wrap}}}}}(text{[}, width{]})
&
Wrap a single paragraph of text, returning a list of wrapped lines.
\\
<% if sphinx_version >= (5, 3) %>\sphinxbottomrule<% else %>\hline<% endif %>
\end{longtable}<% if sphinx_version >= (5, 3) %>
\sphinxtableafterendhook
<% endif %>\sphinxatlongtableend<% if sphinx_version >= (5, 3) %>
<% endif %>\end{savenotes}
\index{TextWrapper (class in textwrap)@\spxentry{TextWrapper}\spxextra{class in textwrap}}

\begin{fulllineitems}
\phantomsection\label{\detokenize{index:textwrap.TextWrapper}}
<%- if sphinx_version > (4, 5) %>
\pysigstartsignatures
<% endif -%>
\pysiglinewithargsret{\sphinxbfcode{\sphinxupquote{class<% if sphinx_version >= (4, 3) %>\DUrole{w<< sphinx6x_2nd('w') >>}{  }<% else %> <% endif %>}}\sphinxcode{\sphinxupquote{textwrap.}}\sphinxbfcode{\sphinxupquote{TextWrapper}}}{\<< sphinxparam >>{\DUrole{n<< sphinx6x_2nd('n') >>}{width}\DUrole{o<< sphinx6x_2nd('o') >>}{=}\DUrole{default_value}{70}}<<sphinxparamcomma>>{\DUrole{n<< sphinx6x_2nd('n') >>}{initial\_indent}\DUrole{o<< sphinx6x_2nd('o') >>}{=}\DUrole{default_value}{\textquotesingle{}\textquotesingle{}}}<<sphinxparamcomma>>{\DUrole{n<< sphinx6x_2nd('n') >>}{subsequent\_indent}\DUrole{o<< sphinx6x_2nd('o') >>}{=}\DUrole{default_value}{\textquotesingle{}\textquotesingle{}}}<<sphinxparamcomma>>{\DUrole{n<< sphinx6x_2nd('n') >>}{expand\_tabs}\DUrole{o<< sphinx6x_2nd('o') >>}{=}\DUrole{default_value}{True}}<<sphinxparamcomma>>{\DUrole{n<< sphinx6x_2nd('n') >>}{replace\_whitespace}\DUrole{o<< sphinx6x_2nd('o') >>}{=}\DUrole{default_value}{True}}<<sphinxparamcomma>>{\DUrole{n<< sphinx6x_2nd('n') >>}{fix\_sentence\_endings}\DUrole{o<< sphinx6x_2nd('o') >>}{=}\DUrole{default_value}{False}}<<sphinxparamcomma>>{\DUrole{n<< sphinx6x_2nd('n') >>}{break\_long\_words}\DUrole{o<< sphinx6x_2nd('o') >>}{=}\DUrole{default_value}{True}}<<sphinxparamcomma>>{\DUrole{n<< sphinx6x_2nd('n') >>}{drop\_whitespace}\DUrole{o<< sphinx6x_2nd('o') >>}{=}\DUrole{default_value}{True}}<<sphinxparamcomma>>{\DUrole{n<< sphinx6x_2nd('n') >>}{break\_on\_hyphens}\DUrole{o<< sphinx6x_2nd('o') >>}{=}\DUrole{default_value}{True}}<<sphinxparamcomma>>{\DUrole{n<< sphinx6x_2nd('n') >>}{tabsize}\DUrole{o<< sphinx6x_2nd('o') >>}{=}\DUrole{default_value}{8}}<<sphinxparamcomma>>{\DUrole{o<< sphinx6x_2nd('o') >>}{*}}<<sphinxparamcomma>>{\DUrole{n<< sphinx6x_2nd('n') >>}{max\_lines}\DUrole{o<< sphinx6x_2nd('o') >>}{=}\DUrole{default_value}{None}}<<sphinxparamcomma>>{\DUrole{n<< sphinx6x_2nd('n') >>}{placeholder}\DUrole{o<< sphinx6x_2nd('o') >>}{=}\DUrole{default_value}{\textquotesingle{} {[}...{]}\textquotesingle{}}}}{}
<% if sphinx_version > (4, 5) %>\pysigstopsignatures
<% endif -%>
Object for wrapping/filling text.  The public interface consists of
the wrap() and fill() methods; the other methods are just there for
subclasses to override in order to tweak the default behaviour.
If you want to completely replace the main wrapping algorithm,
you’ll probably have to override \_wrap\_chunks().
\begin{description}
<% if sphinx_version >= (5, 0) %>\sphinxlineitem<% else %>\item[<% endif %>{Several instance attributes control various aspects of wrapping:}<% if sphinx_version < (5, 0) %>] \leavevmode<% endif %>\begin{description}
<% if sphinx_version >= (5, 0) %>\sphinxlineitem<% else %>\item[<% endif %>{width (default: 70)}<% if sphinx_version < (5, 0) %>] \leavevmode<% endif %>
the maximum width of wrapped lines (unless break\_long\_words
is false)

<% if sphinx_version >= (5, 0) %>\sphinxlineitem<% else %>\item[<% endif %>{initial\_indent (default: “”)}<% if sphinx_version < (5, 0) %>] \leavevmode<% endif %>
string that will be prepended to the first line of wrapped
output.  Counts towards the line’s width.

<% if sphinx_version >= (5, 0) %>\sphinxlineitem<% else %>\item[<% endif %>{subsequent\_indent (default: “”)}<% if sphinx_version < (5, 0) %>] \leavevmode<% endif %>
string that will be prepended to all lines save the first
of wrapped output; also counts towards each line’s width.

<% if sphinx_version >= (5, 0) %>\sphinxlineitem<% else %>\item[<% endif %>{expand\_tabs (default: true)}<% if sphinx_version < (5, 0) %>] \leavevmode<% endif %>
Expand tabs in input text to spaces before further processing.
Each tab will become 0 .. ‘tabsize’ spaces, depending on its position
in its line.  If false, each tab is treated as a single character.

<% if sphinx_version >= (5, 0) %>\sphinxlineitem<% else %>\item[<% endif %>{tabsize (default: 8)}<% if sphinx_version < (5, 0) %>] \leavevmode<% endif %>
Expand tabs in input text to 0 .. ‘tabsize’ spaces, unless
‘expand\_tabs’ is false.

<% if sphinx_version >= (5, 0) %>\sphinxlineitem<% else %>\item[<% endif %>{replace\_whitespace (default: true)}<% if sphinx_version < (5, 0) %>] \leavevmode<% endif %>
Replace all whitespace characters in the input text by spaces
after tab expansion.  Note that if expand\_tabs is false and
replace\_whitespace is true, every tab will be converted to a
single space!

<% if sphinx_version >= (5, 0) %>\sphinxlineitem<% else %>\item[<% endif %>{fix\_sentence\_endings (default: false)}<% if sphinx_version < (5, 0) %>] \leavevmode<% endif %>
Ensure that sentence\sphinxhyphen{}ending punctuation is always followed
by two spaces.  Off by default because the algorithm is
(unavoidably) imperfect.

<% if sphinx_version >= (5, 0) %>\sphinxlineitem<% else %>\item[<% endif %>{break\_long\_words (default: true)}<% if sphinx_version < (5, 0) %>] \leavevmode<% endif %>
Break words longer than ‘width’.  If false, those words will not
be broken, and some lines might be longer than ‘width’.

<% if sphinx_version >= (5, 0) %>\sphinxlineitem<% else %>\item[<% endif %>{break\_on\_hyphens (default: true)}<% if sphinx_version < (5, 0) %>] \leavevmode<% endif %>
Allow breaking hyphenated words. If true, wrapping will occur
preferably on whitespaces and right after hyphens part of
compound words.

<% if sphinx_version >= (5, 0) %>\sphinxlineitem<% else %>\item[<% endif %>{drop\_whitespace (default: true)}<% if sphinx_version < (5, 0) %>] \leavevmode<% endif %>
Drop leading and trailing whitespace from lines.

<% if sphinx_version >= (5, 0) %>\sphinxlineitem<% else %>\item[<% endif %>{max\_lines (default: None)}<% if sphinx_version < (5, 0) %>] \leavevmode<% endif %>
Truncate wrapped lines.

<% if sphinx_version >= (5, 0) %>\sphinxlineitem<% else %>\item[<% endif %>{placeholder (default: ‘ {[}…{]}’)}<% if sphinx_version < (5, 0) %>] \leavevmode<% endif %>
Append to the last line of truncated text.

\end{description}

\end{description}


\vspace{10px}


\sphinxstylestrong{Methods:}


\vspace{-5px}



\begin{savenotes}<% if sphinx_version >= (5, 3) %>
<% endif %>\sphinxatlongtablestart<% if sphinx_version >= (5, 3) %>
\sphinxthistablewithglobalstyle
\sphinxthistablewithnovlinesstyle
<% endif %><% if sphinx_version >= (6, 2) %>\makeatletter
  \LTleft \@totalleftmargin plus1fill
  \LTright\dimexpr\columnwidth-\@totalleftmargin-\linewidth\relax plus1fill
\makeatother
\begin{longtable}{\Xx{4}{10}\Xx{60}{100}}<% else %>\begin{longtable}[c]{\Xx{4}{10}\Xx{60}{100}}<% endif %><% if sphinx_version >= (5, 3) %>
\sphinxtoprule
<% else %>
\hline

<% endif %>\endfirsthead

\multicolumn{2}{c}<% if sphinx_version >= (5, 3) %>{\sphinxnorowcolor
    \makebox[0pt]<% else %>%
{\makebox[0pt]<% endif %>{\sphinxtablecontinued{\tablename\ \thetable{} \textendash{} continued from previous page}}<% if sphinx_version >= (5, 3) %>%
<% endif %>}\\
<% if sphinx_version >= (5, 3) %>\sphinxtoprule
<% else %>\hline

<% endif %>\endhead

<% if sphinx_version >= (5, 3) %>\sphinxbottomrule<% else %>\hline<% endif%>
\multicolumn{2}{r}{<% if sphinx_version >= (5, 3) %>\sphinxnorowcolor
    <% endif %>\makebox[0pt][r]{\sphinxtablecontinued{continues on next page}}<% if sphinx_version >= (5, 3) %>%
<% endif %>}\\
\endfoot

\endlastfoot<% if sphinx_version >= (5, 3) %>
\sphinxtableatstartofbodyhook<% endif%>

{\hyperref[\detokenize{index:textwrap.TextWrapper.fill}]{\sphinxcrossref{\sphinxcode{\sphinxupquote{fill}}}}}(text)
&
Reformat the single paragraph in <<open_smart_quote_b>>text<<close_smart_quote_b>> to fit in lines of no more than <<open_smart_quote_b>>self.width<<close_smart_quote_b>> columns, and return a new string containing the entire wrapped paragraph.
\\
<< hline >>
{\hyperref[\detokenize{index:textwrap.TextWrapper.wrap}]{\sphinxcrossref{\sphinxcode{\sphinxupquote{wrap}}}}}(text)
&
Reformat the single paragraph in <<open_smart_quote_b>>text<<close_smart_quote_b>> so it fits in lines of no more than <<open_smart_quote_b>>self.width<<close_smart_quote_b>> columns, and return a list of wrapped lines.
\\
<% if sphinx_version >= (5, 3) %>\sphinxbottomrule<% else %>\hline<% endif %>
\end{longtable}<% if sphinx_version >= (5, 3) %>
\sphinxtableafterendhook
<% endif %>\sphinxatlongtableend<% if sphinx_version >= (5, 3) %>
<% endif %>\end{savenotes}
\index{fill() (textwrap.TextWrapper method)@\spxentry{fill()}\spxextra{textwrap.TextWrapper method}}

\begin{fulllineitems}
\phantomsection\label{\detokenize{index:textwrap.TextWrapper.fill}}
<%- if sphinx_version > (4, 5) %>
\pysigstartsignatures
<% endif -%>
\pysiglinewithargsret{\sphinxbfcode{\sphinxupquote{fill}}}{\<< sphinxparam >>{\DUrole{n<< sphinx6x_2nd('n') >>}{text}\DUrole{p<< sphinx6x_2nd('p') >>}{:}<% if sphinx_version >= (4, 3) %>\DUrole{w<<sphinx6x_2nd("w")>>}{  }<% else %> <% endif %>\DUrole{n<<sphinx6x_2nd("n")>>}{
<%- if sphinx_version >= (4, 4) %>string}}}{{ $\rightarrow$ string}}
<%- else %>{\hyperref[\detokenize{index:module-string}]{\sphinxcrossref{string}}}}}}{{ $\rightarrow$ {\hyperref[\detokenize{index:module-string}]{\sphinxcrossref{string}}}}}
<%- endif %>
<%- if sphinx_version > (4, 5) %>
\pysigstopsignatures
<%- endif %>
Reformat the single paragraph in ‘text’ to fit in lines of no
more than ‘self.width’ columns, and return a new string
containing the entire wrapped paragraph.

\end{fulllineitems}

\index{wrap() (textwrap.TextWrapper method)@\spxentry{wrap()}\spxextra{textwrap.TextWrapper method}}

\begin{fulllineitems}
\phantomsection\label{\detokenize{index:textwrap.TextWrapper.wrap}}
<%- if sphinx_version > (4, 5) %>
\pysigstartsignatures
<% endif -%>
\pysiglinewithargsret{\sphinxbfcode{\sphinxupquote{wrap}}}{\<< sphinxparam >>{\DUrole{n<< sphinx6x_2nd('n') >>}{text}\DUrole{p<<sphinx6x_2nd("p")>>}{:}<% if sphinx_version >= (4, 3) %>\DUrole{w<<sphinx6x_2nd("w")>>}{  }<% else %> <% endif %>\DUrole{n<<sphinx6x_2nd("n")>>}{
<%- if sphinx_version >= (4, 4) -%>
string}}}{{ $\rightarrow$ \DUrole{p<<sphinx6x_2nd("p")>>}{{[}}string\DUrole{p<<sphinx6x_2nd("p")>>}{{]}}}}
<%- else -%>
{\hyperref[\detokenize{index:module-string}]{\sphinxcrossref{string}}}}}}{{ $\rightarrow$ \DUrole{p<<sphinx6x_2nd("p")>>}{{[}}{\hyperref[\detokenize{index:module-string}]{\sphinxcrossref{string}}}\DUrole{p}{{]}}}}
<%- endif -%>
<%- if sphinx_version > (4, 5) %>
\pysigstopsignatures
<%- endif %>
Reformat the single paragraph in ‘text’ so it fits in lines of
no more than ‘self.width’ columns, and return a list of wrapped
lines.  Tabs in ‘text’ are expanded with string.expandtabs(),
and all other whitespace characters (including newline) are
converted to space.

\end{fulllineitems}


\end{fulllineitems}

\index{dedent() (in module textwrap)@\spxentry{dedent()}\spxextra{in module textwrap}}

\begin{fulllineitems}
\phantomsection\label{\detokenize{index:textwrap.dedent}}
<%- if sphinx_version > (4, 5) %>
\pysigstartsignatures
<% endif -%>
\pysiglinewithargsret{\sphinxcode{\sphinxupquote{textwrap.}}\sphinxbfcode{\sphinxupquote{dedent}}}{\<<sphinxparam>>{\DUrole{n<< sphinx6x_2nd('n') >>}{text}}}{}
<% if sphinx_version > (4, 5) %>\pysigstopsignatures
<% endif -%>
Remove any common leading whitespace from every line in \sphinxtitleref{text}.

This can be used to make triple\sphinxhyphen{}quoted strings line up with the left
edge of the display, while still presenting them in the source code
in indented form.

Note that tabs and spaces are both treated as whitespace, but they
are not equal: the lines << '”' if docutils_version < (0, 19) else '“' >>  hello” and “thello” are
considered to have no common leading whitespace.

Entirely blank lines are normalized to a newline character.

\end{fulllineitems}

\index{fill() (in module textwrap)@\spxentry{fill()}\spxextra{in module textwrap}}

\begin{fulllineitems}
\phantomsection\label{\detokenize{index:textwrap.fill}}
<%- if sphinx_version > (4, 5) %>
\pysigstartsignatures
<% endif -%>
\pysiglinewithargsret{\sphinxcode{\sphinxupquote{textwrap.}}\sphinxbfcode{\sphinxupquote{fill}}}{\<<sphinxparam>>{\DUrole{n<< sphinx6x_2nd('n') >>}{text}}<<sphinxparamcomma>>{\DUrole{n<< sphinx6x_2nd('n') >>}{width}\DUrole{o<< sphinx6x_2nd('o') >>}{=}\DUrole{default_value}{70}}<<sphinxparamcomma>>{\DUrole{o<< sphinx6x_2nd('o') >>}{**}\DUrole{n<< sphinx6x_2nd('n') >>}{kwargs}}}{}
<% if sphinx_version > (4, 5) %>\pysigstopsignatures
<% endif -%>
Fill a single paragraph of text, returning a new string.

Reformat the single paragraph in ‘text’ to fit in lines of no more
than ‘width’ columns, and return a new string containing the entire
wrapped paragraph.  As with wrap(), tabs are expanded and other
whitespace characters converted to space.  See TextWrapper class for
available keyword args to customize wrapping behaviour.

\end{fulllineitems}

\index{indent() (in module textwrap)@\spxentry{indent()}\spxextra{in module textwrap}}

\begin{fulllineitems}
\phantomsection\label{\detokenize{index:textwrap.indent}}
<%- if sphinx_version > (4, 5) %>
\pysigstartsignatures
<% endif -%>
\pysiglinewithargsret{\sphinxcode{\sphinxupquote{textwrap.}}\sphinxbfcode{\sphinxupquote{indent}}}{\<<sphinxparam>>{\DUrole{n<< sphinx6x_2nd('n') >>}{text}}<<sphinxparamcomma>>{\DUrole{n<< sphinx6x_2nd('n') >>}{prefix}}<<sphinxparamcomma>>{\DUrole{n<< sphinx6x_2nd('n') >>}{predicate}\DUrole{o<< sphinx6x_2nd('o') >>}{=}\DUrole{default_value}{None}}}{}
<% if sphinx_version > (4, 5) %>\pysigstopsignatures
<% endif -%>
Adds <<open_smart_quote>>prefix<<close_smart_quote>> to the beginning of selected lines in ‘text’.

If ‘predicate’ is provided, ‘prefix’ will only be added to the lines
where ‘predicate(line)’ is True. If ‘predicate’ is not provided,
it will default to adding ‘prefix’ to all non\sphinxhyphen{}empty lines that do not
consist solely of whitespace characters.

\end{fulllineitems}

\index{shorten() (in module textwrap)@\spxentry{shorten()}\spxextra{in module textwrap}}

\begin{fulllineitems}
\phantomsection\label{\detokenize{index:textwrap.shorten}}
<%- if sphinx_version > (4, 5) %>
\pysigstartsignatures
<% endif -%>
\pysiglinewithargsret{\sphinxcode{\sphinxupquote{textwrap.}}\sphinxbfcode{\sphinxupquote{shorten}}}{\<<sphinxparam>>{\DUrole{n<< sphinx6x_2nd('n') >>}{text}}<<sphinxparamcomma>>{\DUrole{n<< sphinx6x_2nd('n') >>}{width}}<<sphinxparamcomma>>{\DUrole{o<< sphinx6x_2nd('o') >>}{**}\DUrole{n<< sphinx6x_2nd('n') >>}{kwargs}}}{}
<% if sphinx_version > (4, 5) %>\pysigstopsignatures
<% endif -%>
Collapse and truncate the given text to fit in the given width.

The text first has its whitespace collapsed.  If it then fits in
the \sphinxstyleemphasis{width}, it is returned as is.  Otherwise, as many words
as possible are joined and then the placeholder is appended:

\begin{sphinxVerbatim}[commandchars=\\\{\}]
\PYG{g+gp}{\PYGZgt{}\PYGZgt{}\PYGZgt{} }\PYG{n}{textwrap}\PYG{o}{.}\PYG{n}{shorten}\PYG{p}{(}\PYG{l+s+s2}{\PYGZdq{}}\PYG{l+s+s2}{Hello  world!}\PYG{l+s+s2}{\PYGZdq{}}\PYG{p}{,} \PYG{n}{width}\PYG{o}{=}\PYG{l+m+mi}{12}\PYG{p}{)}
\PYG{g+go}{\PYGZsq{}Hello world!\PYGZsq{}}
\PYG{g+gp}{\PYGZgt{}\PYGZgt{}\PYGZgt{} }\PYG{n}{textwrap}\PYG{o}{.}\PYG{n}{shorten}\PYG{p}{(}\PYG{l+s+s2}{\PYGZdq{}}\PYG{l+s+s2}{Hello  world!}\PYG{l+s+s2}{\PYGZdq{}}\PYG{p}{,} \PYG{n}{width}\PYG{o}{=}\PYG{l+m+mi}{11}\PYG{p}{)}
\PYG{g+go}{\PYGZsq{}Hello [...]\PYGZsq{}}
\end{sphinxVerbatim}

\end{fulllineitems}

\index{wrap() (in module textwrap)@\spxentry{wrap()}\spxextra{in module textwrap}}

\begin{fulllineitems}
\phantomsection\label{\detokenize{index:textwrap.wrap}}
<%- if sphinx_version > (4, 5) %>
\pysigstartsignatures
<% endif -%>
\pysiglinewithargsret{\sphinxcode{\sphinxupquote{textwrap.}}\sphinxbfcode{\sphinxupquote{wrap}}}{\<<sphinxparam>>{\DUrole{n<< sphinx6x_2nd('n') >>}{text}}<<sphinxparamcomma>>{\DUrole{n<< sphinx6x_2nd('n') >>}{width}\DUrole{o<< sphinx6x_2nd('o') >>}{=}\DUrole{default_value}{70}}<<sphinxparamcomma>>{\DUrole{o<< sphinx6x_2nd('o') >>}{**}\DUrole{n<< sphinx6x_2nd('n') >>}{kwargs}}}{}
<% if sphinx_version > (4, 5) %>\pysigstopsignatures
<% endif -%>
Wrap a single paragraph of text, returning a list of wrapped lines.

Reformat the single paragraph in ‘text’ so it fits in lines of no
more than ‘width’ columns, and return a list of wrapped lines.  By
default, tabs in ‘text’ are expanded with string.expandtabs(), and
all other whitespace characters (including newline) are converted to
space.  See TextWrapper class for available keyword args to customize
wrapping behaviour.

\end{fulllineitems}



\renewcommand{\indexname}{Python Module Index}
\begin{sphinxtheindex}
\let\bigletter\sphinxstyleindexlettergroup
\bigletter{s}
\item\relax\sphinxstyleindexentry{string}\sphinxstyleindexpageref{index:\detokenize{module-string}}
\indexspace
\bigletter{t}
\item\relax\sphinxstyleindexentry{textwrap}\sphinxstyleindexpageref{index:\detokenize{module-textwrap}}
\end{sphinxtheindex}

\renewcommand{\indexname}{Index}
\printindex
\end{document}
